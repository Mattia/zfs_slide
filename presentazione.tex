\documentclass{beamer}
\usetheme{AnnArbor}
\title[COW Filesystem]{Copy On Write Filesystem come strumenti di backup incrementale}
\author[Mattia Valzelli]{Mattia Valzelli}
\institute[Univr]{Universit\`{a} degli studi di Verona}
\date{11 Luglio 2013}
\begin{document}
\begin{frame}
\titlepage
\end{frame}
\section{Introduzione} %Section e' il titolo in alto (riga piccola)
\begin{frame} %frame e' una nuova slide
\frametitle{Perch\'e studiare i filesystem moderni} %titolo della slide
I filesystem moderni sono una parte di tecnologia che ha raggiunto il limite della sua curva.
\end{frame}

\section{Cosa \'e un filesystem}
\begin{frame}
Un filesystem \'e un'astrazione della rappresentazione dei dati su disco.
\end{frame}

\section{Problema con i filesystem moderni}
\begin{frame}
I filesystem moderni sono piccole evoluzioni incrementali da quelli degli anni 90.
Come faccio a fare un backup?
Backup esterni o interni
\end{frame}

\section{Le soluzioni disponibili}
\begin{frame}
\frametitle{Microsoft}
NTFS e porcate, salvataggio solo su disco locale. Parliamoci chiaro, funziona sostanzialmente male.
\end{frame}
\begin{frame}
\frametitle{Linux}
Ubuntu ha una sorta di programma di backup, ma ancora acerbo. Linux nel kernel non ha molto, e comunque ancora meno per l'utente comune (senza riga di comando non vado da nessuna parte).
\end{frame}

\begin{frame}
\frametitle{Apple}
Ha quello piu' semplice e utilizzabile da chiunque (con grafica accattivante). Solo backup esterno. Device apposito per backup con quello che si avvicina a quello che vorrei vedere, comunque).
\end{frame}

\section{ZFS}
\begin{frame}
\frametitle{ZFS}
Piccola introduzione su ZFS, concetto di filesystem COW.
\end{frame}

\begin{frame}
\frametitle{Snapshot e Clone}
Concetto di snapshot e clone.
\end{frame}

\begin{frame}
\frametitle{ZFS send and receive}
concetto zfs send e receive
\end{frame}

\begin{frame}
\frametitle{combinare il tutto}
come combinare queste cose e avere una soluzione completa.
\end{frame}



\end{document}
